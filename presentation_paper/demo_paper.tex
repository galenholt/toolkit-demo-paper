% Options for packages loaded elsewhere
\PassOptionsToPackage{unicode}{hyperref}
\PassOptionsToPackage{hyphens}{url}
\PassOptionsToPackage{dvipsnames,svgnames,x11names}{xcolor}
%
\documentclass[
  number]{elsarticle}

\usepackage{amsmath,amssymb}
\usepackage{iftex}
\ifPDFTeX
  \usepackage[T1]{fontenc}
  \usepackage[utf8]{inputenc}
  \usepackage{textcomp} % provide euro and other symbols
\else % if luatex or xetex
  \usepackage{unicode-math}
  \defaultfontfeatures{Scale=MatchLowercase}
  \defaultfontfeatures[\rmfamily]{Ligatures=TeX,Scale=1}
\fi
\usepackage{lmodern}
\ifPDFTeX\else  
    % xetex/luatex font selection
\fi
% Use upquote if available, for straight quotes in verbatim environments
\IfFileExists{upquote.sty}{\usepackage{upquote}}{}
\IfFileExists{microtype.sty}{% use microtype if available
  \usepackage[]{microtype}
  \UseMicrotypeSet[protrusion]{basicmath} % disable protrusion for tt fonts
}{}
\makeatletter
\@ifundefined{KOMAClassName}{% if non-KOMA class
  \IfFileExists{parskip.sty}{%
    \usepackage{parskip}
  }{% else
    \setlength{\parindent}{0pt}
    \setlength{\parskip}{6pt plus 2pt minus 1pt}}
}{% if KOMA class
  \KOMAoptions{parskip=half}}
\makeatother
\usepackage{xcolor}
\setlength{\emergencystretch}{3em} % prevent overfull lines
\setcounter{secnumdepth}{5}
% Make \paragraph and \subparagraph free-standing
\ifx\paragraph\undefined\else
  \let\oldparagraph\paragraph
  \renewcommand{\paragraph}[1]{\oldparagraph{#1}\mbox{}}
\fi
\ifx\subparagraph\undefined\else
  \let\oldsubparagraph\subparagraph
  \renewcommand{\subparagraph}[1]{\oldsubparagraph{#1}\mbox{}}
\fi


\providecommand{\tightlist}{%
  \setlength{\itemsep}{0pt}\setlength{\parskip}{0pt}}\usepackage{longtable,booktabs,array}
\usepackage{calc} % for calculating minipage widths
% Correct order of tables after \paragraph or \subparagraph
\usepackage{etoolbox}
\makeatletter
\patchcmd\longtable{\par}{\if@noskipsec\mbox{}\fi\par}{}{}
\makeatother
% Allow footnotes in longtable head/foot
\IfFileExists{footnotehyper.sty}{\usepackage{footnotehyper}}{\usepackage{footnote}}
\makesavenoteenv{longtable}
\usepackage{graphicx}
\makeatletter
\def\maxwidth{\ifdim\Gin@nat@width>\linewidth\linewidth\else\Gin@nat@width\fi}
\def\maxheight{\ifdim\Gin@nat@height>\textheight\textheight\else\Gin@nat@height\fi}
\makeatother
% Scale images if necessary, so that they will not overflow the page
% margins by default, and it is still possible to overwrite the defaults
% using explicit options in \includegraphics[width, height, ...]{}
\setkeys{Gin}{width=\maxwidth,height=\maxheight,keepaspectratio}
% Set default figure placement to htbp
\makeatletter
\def\fps@figure{htbp}
\makeatother

\makeatletter
\makeatother
\makeatletter
\makeatother
\makeatletter
\@ifpackageloaded{caption}{}{\usepackage{caption}}
\AtBeginDocument{%
\ifdefined\contentsname
  \renewcommand*\contentsname{Table of contents}
\else
  \newcommand\contentsname{Table of contents}
\fi
\ifdefined\listfigurename
  \renewcommand*\listfigurename{List of Figures}
\else
  \newcommand\listfigurename{List of Figures}
\fi
\ifdefined\listtablename
  \renewcommand*\listtablename{List of Tables}
\else
  \newcommand\listtablename{List of Tables}
\fi
\ifdefined\figurename
  \renewcommand*\figurename{Figure}
\else
  \newcommand\figurename{Figure}
\fi
\ifdefined\tablename
  \renewcommand*\tablename{Table}
\else
  \newcommand\tablename{Table}
\fi
}
\@ifpackageloaded{float}{}{\usepackage{float}}
\floatstyle{ruled}
\@ifundefined{c@chapter}{\newfloat{codelisting}{h}{lop}}{\newfloat{codelisting}{h}{lop}[chapter]}
\floatname{codelisting}{Listing}
\newcommand*\listoflistings{\listof{codelisting}{List of Listings}}
\makeatother
\makeatletter
\@ifpackageloaded{caption}{}{\usepackage{caption}}
\@ifpackageloaded{subcaption}{}{\usepackage{subcaption}}
\makeatother
\makeatletter
\@ifpackageloaded{tcolorbox}{}{\usepackage[skins,breakable]{tcolorbox}}
\makeatother
\makeatletter
\@ifundefined{shadecolor}{\definecolor{shadecolor}{rgb}{.97, .97, .97}}
\makeatother
\makeatletter
\makeatother
\makeatletter
\makeatother
\ifLuaTeX
  \usepackage{selnolig}  % disable illegal ligatures
\fi
\usepackage[]{natbib}
\bibliographystyle{elsarticle-num}
\IfFileExists{bookmark.sty}{\usepackage{bookmark}}{\usepackage{hyperref}}
\IfFileExists{xurl.sty}{\usepackage{xurl}}{} % add URL line breaks if available
\urlstyle{same} % disable monospaced font for URLs
\hypersetup{
  pdftitle={An integrated toolkit for assessment of hydrology-dependent outcomes in the Murray-Darling Basin: HydroBOT},
  pdfauthor={Galen Holt; Georgia Dwyer; David Robertson; Martin Job; Lara Palmer; Rebecca E Lester},
  pdfkeywords={Murray-Darling Basin, Holistic modeling, Management
modeling, Climate change, Climate adaptation},
  colorlinks=true,
  linkcolor={blue},
  filecolor={Maroon},
  citecolor={Blue},
  urlcolor={Blue},
  pdfcreator={LaTeX via pandoc}}

\setlength{\parindent}{6pt}
\begin{document}

\begin{frontmatter}
\title{An integrated toolkit for assessment of hydrology-dependent
outcomes in the Murray-Darling Basin: HydroBOT}
\author[3]{Galen Holt%
\corref{cor1}%
}
 \ead{galen@deakin.edu.au} 
\author[3]{Georgia Dwyer%
%
}

\author[1]{David Robertson%
%
}

\author[2]{Martin Job%
%
}

\author[2]{Lara Palmer%
%
}

\author[3]{Rebecca E Lester%
%
}


\affiliation[1]{organization={CSIRO},,postcodesep={}}
\affiliation[2]{organization={MDBA},,postcodesep={}}
\affiliation[3]{organization={Deakin University},city={Waurn
Ponds},postcodesep={}}

\cortext[cor1]{Corresponding author}






        





\begin{keyword}
    Murray-Darling Basin \sep Holistic modeling \sep Management
modeling \sep Climate change \sep 
    Climate adaptation
\end{keyword}
\end{frontmatter}
    \ifdefined\Shaded\renewenvironment{Shaded}{\begin{tcolorbox}[sharp corners, breakable, enhanced, boxrule=0pt, borderline west={3pt}{0pt}{shadecolor}, interior hidden, frame hidden]}{\end{tcolorbox}}\fi

\hypertarget{graphical-abstract}{%
\section{Graphical abstract}\label{graphical-abstract}}

(../images/GraphicalAbstract-01.png)\{\#fig-tktomgmt\}

\hypertarget{abstract}{%
\section{Abstract}\label{abstract}}

\emph{WRITE EORDS HERE GEORGIA/GALEN}

FIND.REPLACE for consistent terminology broader ecological values OR
environmental targets ?? ecological objective OR ecological objective
flow OR water availability

historical base level

\hypertarget{introduction}{%
\section{Introduction}\label{introduction}}

Water management in large river systems typically targets a wide range
of desired values. For example, in the Murray-Darling Basin (MDB),
Australia, the objective of water management is to maintain a healthy
working river that supports productive and resilient water-dependent
industries, healthy and resilient ecosystems, and communities with
access to sufficient and reliable water supplies
\citep{murray-darlingbasinauthority2011}. These values, defined here as
any social, economic, environmental or cultural asset or function of
significance, importance, worth, or use, are not unique to Australia;
water is managed to protect these values of communities worldwide
\citep[e.g.][]{stern2019, kaye-blake2014} \citep[\emph{;
\citep{connor2014}}]{ziolkowska2016}. When these values are the target
of water management implies that each value depends on water in some way
but, in many cases, these dependencies are not well-defined and
management proceeds under the relatively simple assumption that, if
water is provided, these values will be maintained.

Water managers typically have a limited range of levers at their
disposal which primarily affect hydrology, e.g.~flow releases from dams
or inundation of wetlands. When non-hydrologic levers are available,
such as water trading rules in the Murray-Darling Basin, their impact is
still often assessed on how they alter hydrologic conditions-- their
impact on the spatio-temporal patterns of flow in the system. Further,
hydrologic modelling is often better-integrated with management
workflows than models of other values. In part, this is due to the
availability of large-scale physical models that provide robust ability
to model flows as they arise from natural drivers, such as rainfall, and
capture infrastructure such as dams and diversions. While complex, these
models have relatively high precision and relatively few outcomes
compared to models of ecological or economic responses, for example. The
impact of particular management actions (e.g.~dam releases) on hydrology
is well-understood and captured by the hydrograph, allowing accurate
targeting of particular aspects of the flow regime by management actions
(e.g. \citep[\emph{; \citep{Loire2021}}]{singer2007}). While these
proximate models of system state and impact of management actions are
invaluable, they do not provide the necessary information to assess the
status of the range of flow-dependent values.

Assessing values as they respond to flow and management actions requires
models of these relationships. Such models exist, though their quality
and extent vary widely, ranging from unstated mental models to
highly-detailed population dynamics or economic models
\citep{lester2019}. When these models are quantitative or computational,
they are written in various languages by subject-matter experts, often
focus on one or a limited suite of \emph{GEORGIA} target values and
return disparate outputs depending on their particular goals and
approaches. These models are often designed with the goal of studying
the dynamics of highly specific variables (e.g.~fish population
responses to water management regimes), not to produce the most useful
analysis for management questions or to capture the breadth of target
values. Moreover, these models cannot individually provide information
about larger-scale values; single-species models are insufficient to
assess the condition of all fish or the whole ecosystem, for example.
Integrating these models into a holistic modelling approach is necessary
to assess management-relevant outcomes of disparate, typically
broad-scale, values under different hydrologic conditions and management
actions. Integrating diverse response models gives each response model
utility beyond its original purpose and identifies where new work is
needed (e.g.~where response models are limited or non-existent). The
creation of an integrated toolkit provides the opportunity for robust
decision-making \citep{harrison2023} and the ability to prioritize water
planning across a range of values and identify conditions that achieve
disproportionately large (or small) impacts, as well as the explicit
consideration of synergies and trade-offs among different
values.\emph{These first few paras probably need more general cites in
them GEORGIA}

Exemplifying these issues, the Murray-Darling Basin Authority (MDBA) is
a federal body charged with water management in the MDB, Australia. The
Murray-Darling Basin is centrally important to the economic, cultural,
social and environmental well being of Australia. Within the MDB, 70\%
of water is utilized for agricultural purposes. The MDB contributes
nearly 2\% to gross domestic product from agriculture and tourism, is
home to 2.2 million people, including more than 40 Aboriginal Nations,
and contains approximately 400,000 water-dependent ecosystems
\citep{hart2021}. While balancing this range of values is difficult in
any highly-utilized basin, the Murray-Darling is unusually dry and
variable compared to other systems of similar importance
\citep{hart2021}. Modern management of the MDB arises from the
Commonwealth \citep{wateract2007}, which established the Murray-Darling
Basin Authority with obligations to develop and implement a Basin Plan
governing water resource management \citep{hart2021} \emph{I'm citing
Hart a lot because it's a big overview, but are there better cites?
Something internal?}.

The MDBA has obligations to manage water to maintain a healthy working
river and the state of the MDB is required to be assessed annually, with
a major review of, and updates to, the Basin Plan made at legislated
intervals \citep[e.g.~15 years between development and first
review,][]{hart2021}. Beyond these requirements, the MDBA has an
increasing focus on assessing system response to climate change and
possible adaptation actions that may be taken by the MDBA itself or
other stakeholders in the MDB \citep{neave2015}. In addition to the
MDBA, Basin states and other stakeholders (e.g.~the Commonwealth
Environmental Water Holder) have additional responsibilities for water
management in the MDB \citep{hart2021} \emph{Is there a better cite for
who has control over what?}. Taken together, these management goals
require a holistic modelling approach that integrates across values and
is adaptable to these various management needs while also simplifying
repeated, ongoing use. These needs are not unique to the MDBA and other
water management groups, researchers and stakeholders are likely to have
similar complex and competing needs.

The MDBA has access to robust hydrologic models describing flows in the
system and responses to current management practices, and these models
are under continual use, development, and improvement ( \emph{What's the
best cite? David, Martin?}). Despite the need for ongoing assessment and
forward planning across a range of values, models of the responses to
those hydrologic conditions are patchy, only represent some values, and
have been developed and used in a more ad-hoc approach rather than
integrated with each other or the hydrologic models. In this paper, we
describe an integrated modelling `toolkit' to address these water
management needs in the Murray-Darling Basin, Australia, hereafter
referred to as \emph{HydroBOT} (Hydrology-dependent Basin Outcomes
Toolkit).

The toolkit described here greatly improves the capacity of the relevant
manager (here, the MDBA) to assess outcomes across a range of values,
provides the structure to adapt and include additional values as
additional component models become available, and addresses a number of
issues with current assessment practices and modelling approaches. By
providing a single, consistent interface to a range of response models,
we avoid the need to manually run models separately and we abstract
their different interfaces, languages, and idiosyncrasies. Moreover, the
toolkit design is highly modular, built to allow the integration of new
response models with limited additional updates to HydroBOT itself.
Continuing with this consistency, HydroBOT provides a standard set of
synthesis approaches and functions that target management-relevant
analysis and interpretation of outcomes from these disparate models with
a common approach and design language. Because the need for assessing
outcomes is nearly universal in water management, and not limited to a
particular scenario or project, HydroBOT is designed with strong
scenario-comparison capabilities but is agnostic to what those scenarios
represent. For a similar reason, the standardised synthesis and outputs
are highly flexible, allowing the user to choose the most relevant
outputs for different management needs.

One common issue with modelling in general, and particularly integrated
models spanning several tools, is that such tools can become a black box
where the relationships modelled are opaque. This can yield mistrust by
the public and other stakeholders influenced by the model, but also
mistrust and misunderstanding by users and developers of the model. To
avoid these issues, HydroBOT has been continuously co-designed with the
relevant management agency, the MDBA, and an emphasis is put on public
code, reproducibility, self-documentation and production of outputs that
describe the model actions (e.g.~causal relationships).

To demonstrate HydroBOT, we develop a set of example scenarios capturing
some qualitative aspects of one intended use: the assessment and
comparison of outcomes under different climate scenarios and different
adaptations to those changes. These axes represent changes to the system
due to processes over which managers have no control (i.e.~a changing
climate), and management actions which would typically be targeted
interventions. We use this demonstration to illustrate the problem space
and show how the toolkit can aid in assessing potential system change
and prioritising management actions to mitigate impacts.

\hypertarget{methods}{%
\section{Methods}\label{methods}}

The purpose of HydroBOT is to assess how hydrographs representing
various climate and adaptation scenarios affect multiple sets of values,
which may span many scales and disciplines. We seek to make these
assessments in a consistent and repeatable way despite differences in
the response models. HydroBOT then has the capability to synthesize a
wide range of outputs from the response models into results that compare
scenarios and are digestible and useful for management decision making
(\textbf{?@fig-tktomgmt}). A co-design process including scientists,
software developers, and managers was developed to ensure the toolkit
achieved its goal of producing scientifically robust results that are
also management-relevant.

\includegraphics{../images/Conceptual_fig_demopaper2.png}\{height = 30
width=30\}

\hypertarget{co-design-and-scoping-for-management-relevance}{%
\subsection{Co-design and scoping for management
relevance}\label{co-design-and-scoping-for-management-relevance}}

Avoiding HydroBOT becoming a `black box' builds trust, interpretability
and usability of outputs created with the toolkit. To ensure HydroBOT
meets management needs and is trusted by management users, a toolkit
development team was created consisting of primary toolkit developers,
hydrologic modelers, and MDBA staff. Collaboration among this team
ensured decisions about toolkit construction reflected the needs of the
end-user. As implementation proceeded, the collaboration provided a
mechanism by which goals could be adjusted or the implications of
decisions discussed and understood. By having this window into the
toolkit development, the managers obtained a much more granular view of
how it operates, and its capabilities and limitations.

Key to the collaborative development was identification of the response
models to include. These models need to provide information about values
relevant to water management decisions as those values respond to
hydrology. Early in the development process, a wide scan was taken to
identify candidate response models. This scan considered models being
used or developed within the MDBA itself, other government agencies
(both federal and state), and externally. This scan found that there
were a number of responses models available, but many were outdated or
highly manual, and their use was patchy both within and across
management agencies \citep{holt2022}. Ultimately, only one was suitably
modern and well-developed to include in HydroBOT, the Environmental
Water Response (EWR) Tool. Other in-development modules were identified
as candidates to include during the life of the toolkit, including
economic and social models. Further, this scan identified values that
are mandated management targets for which there were no available or
in-development response models, highlighting areas needing future work.

\hypertarget{toolkit-overview}{%
\subsection{Toolkit overview}\label{toolkit-overview}}

The key goals of the HydroBOT are to provide a flexible platform for
comparing outcomes between different possible scenarios. Scenario
comparison is essential for use in both evaluating past actions or
conditions and planning for the future, whether to better understand
potential shifts in the system from external forces (e.g.~climate
change) or to assess potential management actions. However it can be
difficult to collapse a multitude of scenarios into a format that is
easily digested by decision-makers. Further, this process often needs to
be supported by feedback from stakeholders, other expert advice and the
social and political landscape. Accordingly, a flexible architecture
with distinct components that can be run separately (modular) or
together can assist with the iterative nature of decisions-making for
Water planning and policy.

\hypertarget{architecture}{%
\subsubsection{Architecture}\label{architecture}}

Taken holistically, the architecture of the toolkit comprises five
primary components and the links between them
(\textbf{?@fig-architecture}). The data flow through the toolkit is
represented by the bold arrows. Input data (hydrographs) is ingested by
the Controller, which packages and runs Response Models. Results from
the Response Models are then processed by the Aggregator, and the
analysis and final outputs are prepared by the Comparer. The two
components outside this data flow are the Causal networks and Spatial
units, which are used within the Aggregator and Comparer to define
groupings for aggregation and for visual representation of space and the
complex inter-relationships defined in the response models. Each of
these components is described in more detail below.

The architecture of the toolkit emphasizes modularity. Each of the
components can save its outputs, allowing users to run either the whole
toolkit or re-run needed components only to update analysis. For
example, if a user wished to change how they aggregate the results of
the modules, they could re-run the aggregation step and the subsequent
comparison step to match. This ability to adjust intermediate steps
allows rapid iteration of results to address a given management
question, or adaptation of preexisting results to new questions. This
modularity also allows rapid, iterative development of the toolkit
itself. Any of the components of the toolkit can be updated without
affecting others, and so obtaining new results from updated components
(e.g.~new aggregation capability) simply requires re-running the toolkit
from that point forward.

As each stage of the toolkit runs, it produces yaml and json metadata
files including the run settings and other attached information relating
to the scenarios to enhance repeatability, ensure correctness and
increase comprehension. These metadata include all parameters for the
run, allowing exactly repeatable analyses to be conducted by using the
metadata files as parameter files for subsequent runs. Even if a run is
started in the middle, such as to re-run aggregation in a different way,
the metadata for that run will capture the metadata for the previous
steps, ensuring that outputs at every stage are tagged with full
provenance information about the run that created them.

\hypertarget{implementation}{%
\subsubsection{Implementation}\label{implementation}}

The toolkit is available as an R package \emph{GIT HUB LINK}, which
provides a suite of functions representing the steps in the
architecture. These functions are designed to be general, allowing users
much flexibility in how they run the toolkit for a particular set of
analyses while retaining a consistent structure and outputs. Because
translating from hydrologic scenarios to various responses is a general
problem in water management, we expect the ways in which the toolkit is
used will be highly variable. Thus, by providing a general structure in
the toolkit, users can target the particular questions and particular
analyses needed for a given question. For example, in some cases we
might want to look at the responses of different components of fish life
cycles for a small subset of locations in the MDB. In another situation,
we might want a big-picture view of how climate might shift long-term
goals for the environment as a whole. Although the initial impetus for
creating the toolkit was to assess climate scenarios, its use in
practice can be far more general. Any hydrograph can be assessed, and
any set of scenarios represented by hydrographs can be compared,
provided a response model exists.

The toolkit is designed for analysis of management questions in the
Murray-Darling Basin, and so along with the functions to perform the
analyses, it also provides a standard set of spatial data comprising the
MDB itself, gauges within the MDB (at which hydrographs may be
available), and various management units (Sustainable Diversion Limit
units, Resource Plan Areas, and catchment boundaries for major
subcatchments). Some example hydrograph data are also included for
testing and demonstration. The toolkit provides a clean causal network
for included modules, described in more detail in
Section~\ref{sec-causal_networks}. Analogous elements can be constructed
for other spatial units or responses to extend functionality within or
outside the Murray-Darling Basin.

As a result of the scan of available response models, toolkit
development proceeded with the EWR tool as a single response model, but
with an architecture designed to allow modular integration of additional
response models as they become available. The scan accentuated a
critical feature of this modularity; the toolkit must be able to
incorporate and standardise models written in various languages and with
a wide range of input needs and outputs. The toolkit achieves this by
wrapping those other tools so as to make their differences as hidden
from the user as possible. In the case of Python modules, the toolkit
uses the \{reticulate\} R package \citep{ushey2023} in combination with
small amounts of internal Python code to call Python modules directly.
The Python code internal to the toolkit performs limited cleaning and
translation to prevent passing large objects between languages and
ensure that any idiosyncrasies in module inputs and outputs are handled
consistently. Python dependencies (and Python itself) are automatically
installed on first use of Controller functions that call Python modules
unless they already exist on the user's system. This approach provides
essentially invisible Python for most users, while providing flexibility
for the user to provide their own Python environment if desired. Modules
in other languages are not yet available, but the key requirement is
that they be available in a format that is scriptable. In that case, the
toolkit will provide small setup and cleanup functions as with the
Python modules and wrapper functions to call these modules.

The modularity of the toolkit means it can be run stepwise, with the
user calling the relevant functions at each step
(\textbf{?@fig-architecture}). The outputs of each step can be saved or
returned directly to an interactive session or both. Typically
sufficient selected outputs would be saved for reproducibility and speed
unless the project is small enough to retain a comprehensive set in
memory or re-run quickly in a notebook. There are also wrapper functions
provided that allow running the entire toolkit from Controller through
Aggregator, which are extremely useful for large runs or remote runs on
batching services. One particularly useful wrapper provides the ability
to run from a yaml config file providing function arguments. This
function allows the use of a default file and a `modified run' file,
making it ideal for holding many parameters constant at a default for a
particular set of analyses and only changing some on a per-run basis in
a smaller file. It also takes command-line or R list arguments, making
it a flexible solution to run the toolkit from the command line, in
scripting contexts, or Quarto notebooks. The metadata saved at each step
in the process is a yaml file with parameters that are a superset of
those needed to run the toolkit. Thus, an exactly identical run can be
produced by running the toolkit using an output metadata file as an
input parameter file. These metadata files also include the git hash,
further allowing reproducibility in the face of code changes.

In practice, the toolkit functions are primarily run in one of two ways.
Interactive investigation of relatively small sets of hydrographs can be
done in notebooks (typically Quarto; \citep{allaire2022}) or simple R
scripts. Larger investigations typically would be run on remote
computers as part of batching systems, whether HPC, Azure, AWS, or
other, with the outputs at the end of the aggregation (and potentially
each step) returned. The Comparer step would usually not be run as part
of this larger batching, but considered interactively for two reasons.
First, through the aggregation step, all operations can proceed in
parallel over scenarios, and in some cases parallelizing over gauges is
possible. The Comparer necessarily looks across scenarios, and so breaks
this parallelisation. Conducting comparisons interactively is typically
not excessively CPU- or memory-intensive, provided the Aggregator step
has been well thought through. If there are enough data to require high
processing or memory, it is unlikely to be simplified enough to make
interpretable figures. Second, the primary goal of the Comparer is to
produce usable, interpretable outputs. Arriving at a set of meaningful
outputs is an iterative process that rarely will be known \emph{a
priori}, and so working through this step interactively is necessary.
If, however, the toolkit is being used for ongoing monitoring of the
same analyses (e.g.~a `dashboard'), then the first iteration may be done
interactively, with subsequent uses incorporating those settings written
into a script to auto-generate the same figures.

\hypertarget{toolkit-components}{%
\subsection{Toolkit components}\label{toolkit-components}}

Here we describe the specific implementation of each component of the
toolkit illustrated in \textbf{?@fig-architecture}. An overview of this
specific implementation is given in Appendix~\ref{tbl-components}, with
descriptions given here.

\hypertarget{sec-controller}{%
\subsubsection{Controller}\label{sec-controller}}

The `Controller' component of the toolkit is the interface between the
externally generated input data (scenarios), the chosen response model,
and other external and internal components of the toolkit
(\textbf{?@fig-architecture}). This component initiates the downstream
processing steps according to user-defined settings for a particular
run. It includes arguments for locating the input data and the response
model(s) to use along with any necessary parameters for those models.
The Controller can also control later components of the toolkit,
allowing the full toolkit to be run at once. These include defining
aggregation steps as discussed in Section~\ref{sec-aggregator} and
analysis of the results with the Comparer ( Section~\ref{sec-comparer}
). The Controller determines whether and where outputs are returned at
each step. Having control over the full toolkit process enables large
batched runs using parameter files to specify the control arguments in
\texttt{yaml} files. This core functionality of the Controller is
delivered via simple functions that apply to the input data for each
scenario and can be looped over scenarios in parallel. The controller
can be accessed by the user by using Quarto notebooks to work
interactively, R scripts, or via the command line, depending on the use
case.

\hypertarget{sec-spatial_data}{%
\subsection{Spatial data}\label{sec-spatial_data}}

\emph{GEORGIA TEXT FOR SOMEWHERE ELSE?} Causal networks are linked
closely to each Response model, and while both are defined externally to
the toolkit they require significant work to integrate into a
compatible, modular toolkit component. Spatial units are typically more
general, typically polygons of management interest, and require only
light changes to make compatible.

TEXT FOR SOMEWHERE ELSE? Not pictured, hydrographs and direct Response
model output can directly utilise the Comparer functionality without
aggregation. Each component of the HydroBOT toolkit is distict, allowing
modular changes to be made without altering the function of other
components.

\hypertarget{sec-causal_networks}{%
\subsection{Causal networks}\label{sec-causal_networks}}

Causal networks are models that describe the topology of dependence
among many drivers and outcomes of different type
\citep{peeters2022})\citep{Martínez2019}. The use of a causal network
framing both at a high level for visualisation and communication with
the MDBA and embedded in HydroBOT illustrate both the goals and the
functioning of the toolkit, avoiding it becoming a black box. The
relevant causal network for water management in the Murray-Darling Basin
captures relationships between management, flow, and outcomes for a wide
range of values, from ecological response to economic performance and
human wellbeing. Thus, they include climatic and management drivers, but
also include the causal relationships which form the basis of the
response models (e.g.~flow timeseries to hydrologic indicators, or
potentially to the life history inherent in a population dynamics
model). They incorporate any links connecting those outcomes to
larger-scale outcomes (e.g.~from hydrologic indicators to ecological
response). Thus, causal networks define an overarching model from
initial inputs (e.g.~rainfall or flows or adaptations) through to all
values of interest, with each link defining a response model or
component of a response model. The specifications of the models
underlying each link are highly variable. These models range from
detailed physical models linking runoff to flow, to simpler ecological
models of environmental water requirements, to simple averages, to
leaving the model unspecified where information about the relationships
is unavailable. Assessing the quality of knowledge around each link
provides a powerful assessment of knowledge deficiencies and uncertainty
in responses.

Causal networks themselves, i.e.~the structure of the links
(relationships) and nodes (state variables) can be derived from many
sources, including empirical studies defining the existence of causal
relationships and expert opinion. HydroBOT provides a causal network for
included modules, where available, to describe how their outputs arise
from hydrology and how they relate to various levels of
management-relevant outcomes. The causal networks enable: 1) visual
representation of the complex inter-relationships between scenario
inputs and outcomes across a range of objectives and 2) assessment of
outcomes aggregated along the value dimension (see
Section~\ref{sec-aggregator}). The former aids transparency, elucidating
the intentions and causal relationships behind the response models and
is a useful device for communication alongside other final outputs. The
latter allows outcomes to be quantified for individual (or sets of)
objectives (e.g.~fish breeding), values (e.g.~native fish), or at the
overarching levels of environmental, cultural, social, or economic
values. This quantification provides a powerful assessment tool and the
ability to identify synergies and trade-offs across interrelated values.

\hypertarget{sec-modules}{%
\subsubsection{Response models}\label{sec-modules}}

The impacts of climate and adaptation options on social, cultural,
environmental, or economic values are estimated based on causal
relationships between drivers (e.g.~hydrology) and responses (e.g.~the
state of values). The response models may exist in many different forms,
ranging from binary achievement of hydrologic indicators to fully
quantitative responses. These tools are expected to be sourced from
existing or in-development models developed by subject-matter experts
and, as such, will be written in different languages and will target
different outcomes. HydroBOT then provides a unified interface and
ongoing analysis and modelling of the response model results.

\hypertarget{sec-ewr}{%
\paragraph{EWR (Environmental Water Requirements)}\label{sec-ewr}}

The EWR tool, which forms the core of this demonstration, is one such
response model, written in Python and in use by the MDBA internally, the
state of New South Wales for water planning and other interested
stakeholders. It models the response of ecological values of the
Murray-Darling Basin founded on hydrologic indicators paired with causal
relationships to ecological values. It holds databases of the
environmental water requirements (EWRs; the indicators) and the volume,
frequency, timing and duration of flows or inundation required to meet
those indicators (Section~\ref{sec-ewr-table}). These indicators were
developed based on hypothesized relationships to the ecological
objectives of the MDB, which protect or enhance environmental assets and
ecological functions that are valued based on ecological significance
\citep{sheldon2024}. The EWR tool itself only provides an assessment of
the hydrologic indicators. The EWR tool assesses whether spatially
explicit flow timeseries data meets each EWR (hydrologic indicator) at
each gauge (illustrated in Figure~\ref{fig-ewr-example}). The precise
definitions for each EWR differ at each gauge, due to the unique
hydrology and channel morphology. For example, a small fresh for 10 days
is required every year, ideally between October and April to meet
indicator EWR SF1 (small fresh 1), but the flow volume defined as a
`small fresh' differs between gauges. For HydroBOT to model
environmental outcomes, we connect the EWR tool results to a specific
causal network defining these links.

\begin{figure}

{\centering \includegraphics{demo_paper_files/figure-pdf/fig-ewr-example-1.pdf}

}

\caption{\label{fig-ewr-example}The magnitude of required EWR are
illustrated at two example gauges (412002 and 419001) on the historical
base level (scenario E1) hydrographs. This does not illustrate the
requred frequency, duration or timing or EWR; see
Section~\ref{sec-ewr-table} for a table.}

\end{figure}

\emph{GALEN should we still be saying that we developed causal
networks?} The EWR tool results are binary responses of hydrologic
indicators and so, to link these to expected ecological responses, we
developed causal networks from the relationships implied in the
Long-Term Watering Plans in setting the EWRs \citep{DPIE2020abc}. These
plans describe how the hydrologic indicators are expected to influence
both proximate and larger-scale ecological outcomes. We extracted these
causal relationships from the Murray-Darling Basin Long Term Watering
Plans (LTWPs), which were developed based on the best available
information from water managers, ecologists, scientific publications,
and analysis of gauged and modelled flows (e.g.
\citep{nswdepartmentofplanningandenvironment2020},
\citep{lobegeiger2022}). This provided a dense network of links across a
range of ecological outcomes at various scales, from the hydrologic
indicators to components of the life cycle of single species to
whole-community outcomes at 10- or 20-year target dates. For example,
for the SF1 indicator described above might contribute to multiple
ecological objectives pertaining to native fish (ecological objectives
NF1-9; e.g.~NF1 = No loss of native fish species), native vegetation
(NV1), and ecological functions (EF1-5). These ecological objectives are
a defined to support the completion of all elements of a life cycle of
an organism or group of organisms (taxonomic or spatial) to achieve a
goal state, condition or characteristic of an environmental asset or
ecological function'' \emph{{[}LTWP2020{]}}. Outcomes for ecological
objectives are then linked in the causal network to five ecological
values (native fish, native vegetation, waterbirds, other species, and
priority ecological functions) and are associated with long-term targets
(5, 10, and 20 year) of the LTWP's management strategies. The chain of
dependencies from EWRs to ecological objectives to long-term targets are
captured in the causal networks as links, with nodes defining the
ecological objectives or long-term targets
(\textbf{?@fig-causal-example}). This structure not only provides a
visual definition of the links in the LTWP, it also enables assessment
of outcomes in direct equivalence to the LTWP's management strategies.

\begin{figure}

{\centering \includegraphics{demo_paper_files/figure-pdf/fig-causal-example-1.pdf}

}

\caption{\label{fig-causal-example-1}a) HydroBOT incorporates causal
networks that describe the ecological objectives for a system. In the
current example these causal networks are extracted from the
Murray-Darling Basin Long Term Watering Plans (LTWPs), which sets
environmental watering requirements (EWRs), ecological objectives, and
long-term targets for key water-dependent plants, waterbirds, fish and
ecological functions. From left to right, the columns here represent EWR
codes, ecological objectives, specific goals, ecological values, and
5-year management targets. The network shown here is for a single gauge
(412002) and Planning Unit, which is the scale at which EWR codes and
ecological objectives are defined. The other levels are defined at
larger spatial scales, but only those that apply to the EWRs present at
this gauge are shown here. b) Subset of the causal network for gauge
419001 relevant to a selection of \emph{lengthsampleenvobj} ecological
objectives. The data are at the gauge scale for the first and second
columns (EWRs and ecological objectives) and at the SDL unit scale for
the third (ecological values). Thus, the final set of nodes contain
information from other gauges that are not shown here for clarity.
Colors represent the log of the relative change in condition from the
baseline for the halving (panel a, scenario A) and doubling (panel b,
scenario I) `climate' scenarios. Large reductions in condition are dark
purple, large improvements are dark green, and no change is white. See
Appendix 3 \emph{sec-causalnetwork-versions} for more detailed causal
network diagrams.}

\end{figure}

\begin{figure}

{\centering \includegraphics{demo_paper_files/figure-pdf/fig-causal-example-2.pdf}

}

\caption{\label{fig-causal-example-2}a) HydroBOT incorporates causal
networks that describe the ecological objectives for a system. In the
current example these causal networks are extracted from the
Murray-Darling Basin Long Term Watering Plans (LTWPs), which sets
environmental watering requirements (EWRs), ecological objectives, and
long-term targets for key water-dependent plants, waterbirds, fish and
ecological functions. From left to right, the columns here represent EWR
codes, ecological objectives, specific goals, ecological values, and
5-year management targets. The network shown here is for a single gauge
(412002) and Planning Unit, which is the scale at which EWR codes and
ecological objectives are defined. The other levels are defined at
larger spatial scales, but only those that apply to the EWRs present at
this gauge are shown here. b) Subset of the causal network for gauge
419001 relevant to a selection of \emph{lengthsampleenvobj} ecological
objectives. The data are at the gauge scale for the first and second
columns (EWRs and ecological objectives) and at the SDL unit scale for
the third (ecological values). Thus, the final set of nodes contain
information from other gauges that are not shown here for clarity.
Colors represent the log of the relative change in condition from the
baseline for the halving (panel a, scenario A) and doubling (panel b,
scenario I) `climate' scenarios. Large reductions in condition are dark
purple, large improvements are dark green, and no change is white. See
Appendix 3 \emph{sec-causalnetwork-versions} for more detailed causal
network diagrams.}

\end{figure}

\hypertarget{module-standardization}{%
\paragraph{Module standardization}\label{module-standardization}}

Each response model will have a distinct set of outputs, reflecting the
captured responses and the structure of the model. When run within
HydroBOT, these outputs are cleaned and processed into standard,
expected formats for further toolkit processing, and metadata is saved.
This enables HydroBOT to provide a consistent, unified home for
disparate response models. The outputs can be saved to disk or retained
in-memory for interactive use, depending on the user's needs. The
outcomes of the response models are then processed by the Aggregator to
enable outputs to be viewed at relevant scales (in time, space or value
dimensions; Figure~\ref{fig-aggregation-dims}).

\hypertarget{sec-aggregator}{%
\subsubsection{Aggregator}\label{sec-aggregator}}

Results from the response models are typically very granular in many
dimensions because the best response models operate near the scale of
the processes being modelled. In many cases, those processes (e.g.~fish
breeding, crop planting) individually occur at small spatial and
temporal scales. Note that this is the scale of the process itself, but
that may be replicated over much larger scales. For example, fish may
breed across the MDB but each female breeds in only one location at a
given time. Moreover, outcomes from response models are typically at
small value scales as well, e.g.~capturing portions of the life cycle of
fish species, rather than an overall outcome for all fish, or
representing planting of particular crops, rather than overall
agricultural output. The consequence is that there are potentially
thousands of different modeled outcomes across time, locations and
values. This plethora of outcomes must then be aggregated to extract
meaning at larger scales and condense the information for digestibility
in management decision making.

Aggregation condenses that information to scales that are useful for
interpretation and planning. Depending on the use, the desired scale(s)
may range from local, short-term responses of fine-grained outcomes to
large spatial scales over longer time periods for high-level outcomes
such as environmental condition (Figure~\ref{fig-aggregation-dims}).
Thus, HydroBOT must achieve a robust, consistent aggregation approach
along three dimensions (time, space, and value), while maintaining the
ability to define those aggregation steps flexibly to meet the needs of
the specific analysis. For example, the EWR tool assesses EWRs at
individual gauges, but outcomes may be desired within areas defined for
management purposes (e.g.~Sustainable Diversion Limit (SDL) units) or at
the basin scale. Likewise, multiple EWRs are required to meet ecological
objectives, of which many are required for each ecological value or
long-term target (see Appendix 1 for glossary \textbf{?@sec-glossary}).
The HydroBOT Aggregator enables scaling from the hydrology at each gauge
to results at any of these scales. HydroBOT provides a standard set of
spatial units for aggregation but can accept any user-supplied polygons
as the spatial aggregation units (Figure~\ref{fig-aggregation-dims}).
Aggregation along the value dimension follows the causal network, which
is supplied by HydroBOT for included tools, though the user can specify
other causal relationships if required.

\begin{figure}

{\centering \includegraphics{../images/Aggregations.png}

}

\caption{\label{fig-aggregation-dims}Aggregation along multiple
dimensions. HydroBOT provides flexible capability to aggregate along
spatial, temporal and value scales. Users can control the sequence of
steps along these axes, with the capability to switch between axes at
different steps.}

\end{figure}

A flexible approach to aggregation is needed as specific dimensions
(time, space and value) and units within those are best combined in
different ways, depending on the meaning of the data and the use of the
final outputs. The choices made in these aggregation steps are critical
to ensure the final values are scientifically justified and capture the
intended meaning. The Aggregator defines a sequence of aggregation steps
to take with the data, specifying the dimensions along which to
aggregate and the aggregation functions to apply at each step. Each step
can be along any dimension and multiple functions can be assigned at
each step. The sequence of aggregation steps can interleave the
dimensions, e.g.~aggregate along the value dimension to an intermediate
level, then aggregate in space, then value again. For example, following
Figure~\ref{fig-aggregation-dims}, the user might want to aggregate from
hydrologic indicator (EWR) to ecological objective to specific goal
(e.g.~species) at each gauge, then aggregate those gauges into a SDL
unit to assess performance of each species over a larger area, followed
by aggregating to ecological values (broad group, e.g.~native fish,
waterbirds, native vegetation, ecosystem function).

The Aggregator allows the user to choose any aggregation function at
each aggregation step, reflecting the need to account for both the
processes being aggregated and the outputs needed for management
decision making. These aggregation functions should be considered
carefully, as they have different meaning both for the processes being
modeled and the interpretation of the outcomes
(\textbf{?@fig-aggregation\_types}). For example, in some situations the
user might want to know the rate at which EWRs pass across some area, or
perhaps the average value of an abundance measure. In this case, a mean
would be appropriate. In other situations, however, a single failure may
be disproportionately important (e.g.~`no loss' requirements), or
perhaps a single passing value is sufficient (e.g.~bird breeding can
occur anywhere in a catchment). These might be captured with a minimum
and maximum, respectively. To address this need for flexible
aggregation, HydroBOT provides a standard set of functions (e.g.~mean,
max, min, and spatially-weighted mean), but the user can also specify
any other aggregation function, including custom-written functions.

The complexity of the potential aggregation sequences and functions
highlights the importance of tracking the provenance of the final values
to understand their meaning. Thus, the Aggregator saves the sequences
and functions applied at each stage to a metadata file that also can
serve as the aggregation parameters for repeatable runs. Moreover, the
output dataset retains the full sequence and functions alongside each
value, ensuring that values are always paired with their provenance and
meaning.

\hypertarget{sec-comparer}{%
\subsubsection{Comparer}\label{sec-comparer}}

The HydroBOT Comparer is designed primarily to make comparisons between
scenarios (typically hydrographs reflecting climate or climate
adaptation in the examples that follow), allowing assessment and
visualization of their differences. This component also provides
generalized capacity to produce plots and other outputs such as tables
using a consistent approach, even when not directly comparing scenarios.
Comparisons are essential to assess the outcomes of various scenarios,
e.g.~the behaviour of the system under different climate regimes or with
different adaptation options. The functions within the Comparer can be
divided into two main categories, those for analysis and those for
plotting. Although in some instances presenting the absolute outcome
values can be useful across scenarios, explicitly calculating
comparisons (e.g.~the absolute or relative difference between scenarios)
provides distinct advantages. Difficulty in accurately simulating a
complex system means that comparisons between scenario outcomes can be
more useful and accurate because any bias in the baseline assumptions
applies to all outcomes and the focus moves from the total level to the
change between scenarios \emph{RL: Can use the same ref you cite in MER
once you find one. GALEN}. The best method for comparing will vary
depending on the quantities being compared and the intended use of the
comparison, so several common default options (differences, relative
change) have been developed along with the flexibility for the user to
define alternatives.

The comparison functions provide the ability to choose a baseline level
for comparison, which may be one of the scenarios, but also may be a
reference dataset or a scalar value. For example, we might want to
compare a set of outcomes for climate scenarios to a `no change' climate
scenario, to historical observations, or to a mean value. Output values
are calculated relative to the defined baseline using either default
functions for the absolute or relative difference, or any other
user-supplied function. These functions can be applied to any dataframe,
but a major advantage of HydroBOT is including them within the plotting
functions. This allows the generation of comparison plots from raw
Aggregator outputs without the need for subsequent data calculations by
the user and avoids potential errors that arise from repeated or
forgotten data transformations in an analysis workflow.

The plotting functions in the Comparer provide capability to present and
visualize comparisons using standardized procedures for all outputs
within a project. Different purposes require different sets of outputs;
for instance, maps are particularly useful for visualizing geographic
patterns, tables and graphs typically provide more precise ability to
assess impacts on values, and timeseries plots are useful for
visualizing climate trajectories. While potential visualizations and
comparisons vary widely depending on the intended use, the Comparer
standardizes data cleaning and processing for each plot, as well as
aesthetics and plotting approaches. This standardization ensures
consistency across plot types and through the project, ensuring values
plotted are robust and interpretable. Key to this standardization is the
internal data cleaning, which allows the raw outputs of the Aggregator
and arguments for the comparisons to be provided to the plotting
functions. By standardizing data cleaning within the Comparer, we avoid
losing information or performing unsupported data manipulations and so
ensure the quality and meaning of the outputs.

\hypertarget{demonstration-scenarios}{%
\subsection{Demonstration scenarios}\label{demonstration-scenarios}}

We demonstrate toolkit functionality for ecological values by using the
existing EWR (Environmental Water Requirements) tool for the response
model (see Section~\ref{sec-modules} for a description of the EWR module
and its associated causal network). Our input data consists of
hypothetical flow timeseries generated from historical hydrographs at 46
gauges. These gauges fall within the Lachlan, Macquarie--Castlereagh,
and Namoi SDL units of the Murray-Darling Basin, Australia
(Figure~\ref{fig-sdl-comparison}A). These SDL units have detailed LTWPs,
and so well-specified EWRs and causal networks.

We develop a set of simple scenarios that capture two sorts of changes
that may be commonly represented in management analyses. First, we
consider scaled flow throughout the period of the hydrograph,
representing overall increases or decreases in flow as might occur from
large-scale climate patterns. Second, we consider short-duration
additions to flow, representing periodic pulses of change as might
happen from targeted interventions. Each scenario characterizes all
water in the system including natural inflows, extraction, and release
of environmental water, yielding a complete hydrograph
(Section~\ref{sec-scenarios} shows a selected subset). These do not
reflect realistic future scenarios but provide an avenue to test and
illustrate the capabilities of HydroBOT.

To scaled flow, we apply nine flow multipliers, ranging from 0.5 to 2.0,
to the historical hydrographs (Table~\ref{tbl-scenarios}). We refer to
these as `climate' scenarios, reflecting a common representation where
entire hydrographs might shift to represent climate change, though these
scenarios are not derived from climate models and do not represent
hypothesized climate change. To achieve pulsed change for each of the
`climate' scenarios, four flow additions were applied including 1) no
addition (baseline), 2) addition of 250 ML/d, 3) addition of 6500 ML/d,
and 4) addition of 12000 ML/d (Table~\ref{tbl-scenarios}). These
additional flows were added throughout the period of September to
December. We refer to these scenarios as `climate adaptations' because
management options are often available in the form of altering water
availability for short time periods through mechanisms like water
releases, though the options here do not represent proposed actions.

For simplicity for this demonstration, we use arithmetic means for all
aggregations except the very first, though we typically present results
only after this first step. The first step gives achievement of EWR
indicators from timing-dependent sub-indicators. In that case, we
consider that, if an EWR is achieved at any time, it is achieved and so
use a maximum for that aggregation. Though we use means throughout,
their meaning changes depending on the level of aggregation. For
simplicity, we assume the value of each outcome represents some
`condition', and the `condition' at a particular level can be assessed
as the mean of the conditions contributing to it. For example, the
condition of ecological objectives might simply be the proportion of
EWRs that contribute to each objective that are achieved. Then, the
condition of ecological values, e.g.~native fish, might be the mean
condition over all the NF1..n objectives, despite those objectives
depending on different numbers and sets of EWRs. This captures the idea
that native fish condition improves when the life-cycle components
captured by those objectives are met, whether it takes 1 or 10 EWRs to
meet an objective.

\hypertarget{results-and-discussion}{%
\section{Results and discussion}\label{results-and-discussion}}

Here, we present example results produced using HydroBOT, the
demonstration scenarios (described above in Section~\ref{sec-scenarios})
and the EWR tool for the response model. These results demonstrate the
flexibility and robustness of the toolkit, which give it the capability
to produce rigorous but interpretable and management-relevant results in
a reproducible way (all code is available at
https://github.com/MDBAuth/toolkit-demo-paper).

Comparison of the direct outputs from the response model (in this
example, pass/fail of EWRs) without any aggregation along the value
dimension gives an overview of hydrologic responses at different
locations (e.g.~gauges) or areas (e.g.~SDL units) to different
scenarios. Such an assessment can identify areas that may be
particularly vulnerable or which potential `adaptation options' might be
most impactful across many outcomes. In Figure~\ref{fig-sdl-comparison},
we present the EWR outcomes for three climate scenarios (A, E and I;
0.5x, no change and 2x the historical base level, respectively) and
three adaptation options (1, 2 and 3; additions of 0, 250 and 6500 ML/d,
respectively).

This example aggregates the EWR outputs directly to the SDL unit, and
shows that they are likely to be affected by the changes in flow
resulting from these changes in `climate' and application of these
`adaptation options'. However, the scenarios affect each SDL unit
differently. The Lachlan has the highest proportion of EWRs achieved
under all scenarios and has relatively consistent increases in EWR
success with the adaptation options (though the options themselves are
not even; 0 to 250 to 6500 ML/d). The Macquarie--Castlereagh is most
sensitive to small increases in additional water, with large jumps
between `adaptation' options 1 and 2. In all situations, the `climate'
scenarios have less of an impact than the `adaptation' scenarios, though
neither is reflective of expected change in these dimensions. These
outcomes are not yet directly linked to ecological values (priority
ecological functions and environmental assets) and so each hydrologic
indicator (EWR) at each gauge is represented with equal importance,
whether it is required for all ecological objectives or just one.

\begin{figure}

{\centering \includegraphics{demo_paper_files/figure-pdf/fig-sdl-comparison-1.pdf}

}

\caption{\label{fig-sdl-comparison}(a) Map of the Murray-Darling basin,
in eastern Australia. (b) Gauge locations within three SDL units
(Macquarie--Castlereagh, Lachlan, and Namoi). Scenario comparison for
different Sustainable Diversion Limit areas (a) emphasizing quantititive
differences and (b) spatial patterns. The mean proportion of
environmental watering requirements (EWRs) that are achieved under each
scenario for each SDL unit are shown. Climate scenarios and adaptation
options as in Table~\ref{tbl-scenarios}.}

\end{figure}

Response models may not themselves provide responses to all values of
interest, and so we use the causal networks to link these outputs to
additional objectives. In the EWR example, this linking is essential to
address ecological outcomes and investigate how changes in flow affect
values of interest. Thus, we can compare how native fish, native
vegetation, waterbirds, other species, and ecological function targets
are likely to respond to our hypothetical set of climates and adaptation
scenarios. We are also able to compare the planning unit areas and the
scenarios themselves. While there are many ways to display this
information, Figure~\ref{fig-obj_in_groups} consolidates a large amount
of information across scenarios, planning units, and multiple scales of
objective. Each set of colours (and row of panels) represents outcomes
at the larger ecological value level. Within those, the different shades
represent single ecological objectives (e.g.~WB1; Maintain the number
and type of waterbird species), with the heights of these shades being
the proportion of the constituent hydrologic requirements that are met
(EWRs). Because the overall bar heights are then a sum of these
proportions, we provide a dashed line as reference that shows the
situation where all EWRs are met for each ecological objective. This
reference is necessary because the number of EWRs contributing to each
ecological objective, and the number of ecological objectives vary
within each larger ecological value group and spatial unit.

This view provides a useful overview of the impact of different
scenarios on the ecological values across space, while also retaining
the ability to assess individual ecological objectives. For example, WB1
(Maintain the number and type of waterbird species) and WB2 (Increase
total waterbird abundance across all functional groups) are unlikely to
be achieved in the Macquarie--Castlereagh under any scenarios, while
they are met in the baseline and 2x climate scenarios (2 and 3) in both
the Lachlan and Namoi. At the larger ecological value scale, we can see
that the effect of halving water (moving from the baseline (2) scenario
to the 1 scenario) has a disproportionately greater impact on native
fish than doubling water (the 3 scenario) in the Lachlan, while these
shifts are more proportional in the Namoi.

\begin{figure}

{\centering \includegraphics{demo_paper_files/figure-pdf/fig-obj_in_groups-1.pdf}

}

\caption{\label{fig-obj_in_groups}Proportion of EWRs achieved for
broader ecological values, including: priority ecological Function (EF),
native fish (NF), native vegetation (NV), other species (OS) and
waterbirds (WB) (Colours \& rows). Columns are scaled to illustrate the
number of ecological objectives that contribute to each broader
ecological value, and shades of colours illustrate the proportion of
EWRs that contribute to each ecological objective. I.e. if all EWR
contributing to ecological objectives for each ecological values the
columns would reach the dashed horizonal lines. This view provides an
assessment of the performance of individual ecological objectives, as
well as how those contribute to the overall sensitivity of the broader
groups. Not all EWR are applicable in all locations and there are not
equal number of EWR in each category. There are no other species
objectives in the Namoi\emph{GALEN/GEORGIA CHECK THIS ONCE RERUN THORUGH
EWR TOOL}. This illustration includes three climate scenarios: A (0.5x),
E (historical base level/no change), and I (2x); and three adaptation
options: 1 (no adaptation), 2 (+250 ML/d), and 3 (+6500 ML/d)}

\end{figure}

\begin{figure}

{\centering \includegraphics{demo_paper_files/figure-pdf/fig-gauge-to-sdl-map-1.pdf}

}

\caption{\label{fig-gauge-to-sdl-map}Ecological objectives aggregated to
SDL unit spatial scale, illustrated here for three examples (Priority
ecological Function 3, native fish 1, and waterbirds 4). Polygon colours
indicate the proportion of EWR passed for each objective in each SDL
unit. Aggregation as in Figure~\ref{fig-spatial-scaling}. Includes three
climate scenarios: A (0.5x), E (historical base level/no change), and I
(2x); and three adaptation options: 1 (no adaptation), 2 (+250 ML/d),
and 3 (+6500 ML/d), though not all combinations see
Appendix~\ref{sec-map-versions}.}

\end{figure}

While we can see which scenarios are better or worse than others with
bar graphs (Figure~\ref{fig-obj_in_groups}) or maps
(Figure~\ref{fig-gauge-to-sdl-map}), they can make it difficult to
accurately assess relative differences that may illuminate
disproportionate impacts and thresholds for when adaption options could
be most effectively adopted. In general, this requires quantitative
x-axes which would typically be referenced to some baseline. Here, we
can represent both our `climate' and `adaptation' scenarios on a single
axis by quantifying the difference in mean flow from the baseline
condition (historical hydrograph, E1) using the baselining functionality
provided by HydroBOT (see Section~\ref{sec-baselining}). Plotting the
proportion of EWRs achieved for each of the native fish ecological
objectives shows how that EWR achievement is related to changes in the
overall levels of flow. One striking feature of
Figure~\ref{fig-difference-baseline} is that the lines do not smoothly
increase; EWR achievement is not simply a direct relationship to flow
volumes. Instead, the timing matters. There are particularly steep
slopes between the `1' (baseline) and `2' (+250 ML/d) adaptations. The
application of water at those particular times is thus
disproportionately impactful, yielding greater outcomes even when they
have less water than the baseline adaptations (`1s') for climate
scenarios `A' and `E'. Such analyses can identify highly efficacious
points to add water to the system.

\begin{figure}

{\centering \includegraphics{demo_paper_files/figure-pdf/fig-difference-baseline-1.pdf}

}

\caption{\label{fig-difference-baseline}Quantitative scenario comparison
for native fish objectives for three SDL units. Scenarios defined here
as the difference in mean flow (GL) from the baseline historical
scenario (E1). A key message here is that the lines are not smooth --
EWR achievement is not simply a function of the amount of water.
Instead, the `adaptation' options, particularly the `2' scenarios of
adding 250 ML/d of water, have disproportionately large impact (steep
slopes). Shading indicates different ecological objectives within the
native fish ecological value. This illustration includes three climate
scenarios: A (0.5x), E (historical base level/no change), and I (2x) and
three adaptation options: 1 (no adaptation), 2 (+250 ML/d), and 3 (+6500
ML/d).}

\end{figure}

With a large number of scenarios, we can better characterise the way
values respond along different axes of change. Here, we use smoothed
fits to show trends in the performance of ecological objectives (and
groups thereof) in response to multiplicative flow changes (`climate'
scenarios) and pulsed flow additions (`adaptations')
(Figure~\ref{fig-smooth-climate-adapt}, Figure~\ref{fig-smooth-all}). We
use baselining (see Section~\ref{sec-baselining}) to show the relative
change on both the x (flow) and y (response) axes. Thus, any slope other
than 1:1 represents a disproportionate shift in condition. For example,
other species and native vegetation show steeper increases with
increasing flow relative to baseline than decreasing flow, while native
fish respond disproportionately to flow change across the range
considered here. Moreover, because we use smoothed fits, we can identify
thresholds or areas of flow change that are disproportionately more
important (small changes in flow can yield large shifts in condition).

\emph{RL: This is confusing without any discussion of what Fig 20 is. Do
we need it here or can this para move to the supplement too?} \emph{GH:
Hmm. I seem to have made major changes to these figs but not the text.
Fig 20 used to be here, and fig 13 used to be different. The idea was
that panel C could be used to accentuate the main point of Fig 20, that
the adaptations here have a way bigger impact than the climate. So, yes,
we should move most of this to the appendix. If we keep C, we should
keep a sentence about it. And maybe we should decide we don't need C at
all and cut it back off the figure.} In Figure~\ref{fig-smooth-all} the
y-axis scales are much larger than the x, and so proportionality is
difficult to assess. However, this figure makes clear the nonlinear
nature of the response to flow shift, particularly for the no adaptation
scenario (1), as well as changes in sensitivity between adaptation
options. The response to adapatation options (differences between the
lines in Figure~\ref{fig-smooth-all}, slope of lines in
Figure~\ref{fig-smooth-climate-adapt} panel b) tends to be much stronger
than the response to climate (slopes of the lines in
Figure~\ref{fig-smooth-all} and Figure~\ref{fig-smooth-climate-adapt}
panel a), even though the shift in total water might be lower. Moreover,
the adaptation options are not as sensitive to the climate scenarios;
adaptation options other than `1' (no adaptation) show less change with
climate in Figure~\ref{fig-smooth-all}. Thus, these `adaptations' are
increasing resilience to those holistic flow changes. Although these
scenarios do not represent true adaptation options or climate scenarios,
this shows that such changes to resilience are possible with targeted
interventions, and HydroBOT provides the capacity to investigate them.

The responses seen in Figure~\ref{fig-smooth-all} and
Figure~\ref{fig-smooth-climate-adapt} vary between SDL unit (columns)
and ecological values, indicating different sensitivity to both
`climate' and `adaptation' flow conditions across space and among
ecological groupings. For example, although both the Lachlan and
Macquarie-Castlereagh have no successful waterbird outcomes under
baseline conditions, targeted flow conditions have a greater ability to
increase waterbird success in the Lachlan.

\begin{figure}

{\centering \includegraphics{demo_paper_files/figure-pdf/fig-smooth-climate-adapt-1.pdf}

}

\caption{\label{fig-smooth-climate-adapt}(a) Smoothed fits of shifts in
condition (proportion of EWR achieved relative to the scenario with no
climate or adaptation changes {[}E1{]}) of broad ecological values to
relative changes in water availability (also relative to E1). All
climate scenarios (without adaptation; adaptation = 1) are included. (b)
The full combination of scenarios (all climate and adaptations options)
are shown for the example of native fish in the Namoi. This illustrates
the larger impact of the `adaptation options' used here compared to the
`climate' shifts. Fitted lines are loess smooths. Separate lines are fit
to each ecological value. Columns are SDL units. Y-axes are restricted
to better visualise the majoity of the data; thus, a small amount of
data is not plotted, but is accounted for in the loess fits. The full
set of SDL units and groups as in c shown in Appendix
Figure~\ref{fig-smooth-all}.}

\end{figure}

Scenarios will often be defined along more than one axis. In the
demonstration here, we define both a flow multiplication axis as a proxy
for `climate' shifts, and a flow addition as a proxy for `adaptations'.
Moreover, it will often be the case that values will respond to several
different aspects of the flow regime, for example the mean flow and the
flow variance or the return interval of floods or low-flow periods.
Heatmaps or contour surfaces provide a powerful tool for visualising
these interacting responses (Figure~\ref{fig-contour},
Figure~\ref{fig-heatmap}). By examining multiple driver axes, such plots
can highlight important interactions and identify where thresholds occur
and where change along one axis (e.g.~adaptations) can mitigate change
along the other (e.g.~climate shifts). This ability to identify the axes
that provide resilience or sensitivity to changes in others will be
critically important for management uses of this toolkit targeting
climate change adaptation. These assessments must also take spatial and
value differences into account; the different ecological values and SDL
units show very different patterns in how they respond to the
interaction between the `climate' and `adaptation' scenarios
(Figure~\ref{fig-contour}).

Perhaps most usefully, if scenarios are carefully developed to explore
the range of potential change in the system, then the heatmap represents
the response surface of values over this range. Developing the response
surface can be iterative, with subsequent scenarios developed to
increase resolution near areas of rapid change in outcome. This approach
differs from the usual approach of assessing only a small set of
scenarios targeting specific proposed actions or environmental
conditions. Instead, by covering the range to generate a response
surface, it can provide hypotheses about how proposed scenarios might
perform and assess how sensitive the outcomes are to uncertainty in
single scenarios. Specific proposed scenarios could of course be
assessed as well and mapped onto the surface. For example, we might
think of the flow multipliers here as defining the entire range of
plausible climate shifts, and the addition scenarios as the entire
potential range of additions. Then, specific proposed adaptations or
climate sequences could be mapped onto these heatmaps at particular
points. The reverse is also possible; identification of areas of high
sensitivity in the heatmaps could be used to choose a set of potential
adaptations designed to increase climate resilience. This idea extends
to additional dimensions, which can then be collapsed in various ways
(e.g.~via principal component analysis, or by identifying dominant
hydrometrics) to visualise with 2d heatmaps even when the response
surface itself is best defined and studied in more dimensions.

\begin{figure}

{\centering \includegraphics{demo_paper_files/figure-pdf/fig-contour-1.pdf}

}

\caption{\label{fig-contour}Condition results visualised as a surface
with the two scenario definitions on the axes. This approach allows
visualising changes in outcome as the result of multiple axes on which
scenarios might differ. These axes might be different aspects of
scenario creation (as here), or they might be different axes describing
the outcome of scenarios (e.g.~two different hydrometrics such as mean
flow and flow variance)}

\end{figure}

Causal networks can be used to show outcomes in addition to the causal
relationships themselves. This sort of visualisation allows
identification of the holistic consequences of the different scenarios
across the complex and interrelated sets of values, as well as
identifying nodes that are unusually resilient or sensitive to change
(\textbf{?@fig-network-subset}). For example, NF7 is little changed
across the range of climate scenarios considered, while EF5 shows
greater sensitivity. The network structure can also identify critical
nodes that have outsize consequences for other nodes in the network.
Such nodes are critically important for outcomes, and so should be
investigated to determine whether they capture key aspects of the system
or are instead an artifact of model structure (possibly resulting from
the way the network was developed) that may bias results. If their
sensitivity does in fact capture the system, these nodes may be ideal
targets for management intervention. The edges also identify the
dependence between values, where we can see that different environmental
outcomes depend on not only different hydrologic indicators (EWRs), but
also different numbers of those indicators.

\hypertarget{values-across-themes}{%
\section{Values across themes}\label{values-across-themes}}

radar plot?

\hypertarget{implications-and-conclusions}{%
\section{Implications and
conclusions}\label{implications-and-conclusions}}

\hypertarget{outline}{%
\subsubsection{Outline}\label{outline}}

What we achieved:

\begin{itemize}
\item
  a toolkit that provides a consistent, scientifically robust and
  repeatable capacity to model responses to flow, with
  management-relevant outputs
\item
  co-design and development between scientists and managers
\end{itemize}

What this provides:

\begin{itemize}
\item
  new ability to assess the wide suite of values the MDBA is mandated to
  manage for.

  \begin{itemize}
  \tightlist
  \item
    moving beyond hydrology
  \end{itemize}
\item
  Clear (or at least open) modelling to build trust
\item
  ability to assess a range of scenarios, including climate and climate
  adaptation
\end{itemize}

Advantages:

\begin{itemize}
\item
  Identification of areas of resilience and sensitivity.
\item
  Uncertainty, understanding parameter space.
\item
  Modularity
\item
  adapt best models for responses into unified model and target
  management relevance
\item
  agnostic to scenario
\item
  causal networks
\end{itemize}

What cool software things do we do?

\begin{itemize}
\item
  Modularity in architecture
\item
  ability to wrap models in multiple languages
\item
  flexibility- e.g.~ability to aggregate differently
\item
  consistent comparison, with data provenance
\item
  metadata and use as parameters for repeatability
\end{itemize}

\hypertarget{from-elsewhere--incorporate}{%
\subsubsection{From elsewhere-
incorporate}\label{from-elsewhere--incorporate}}

\emph{lots of this can be the backbone of the big-picture points above}

Large-scale natural resource management requires the capacity to make
decisions relating to multiple spatial, temporal, and value dimensions,
and is most successful when multiple scales within those dimensions are
considered \citep{moore2021}. HydroBOT is a framework to navigate those
dimensions for water-dependent social, economic, environmental, and
cultural values. Combining such disparate information in a standard and
comparable manner can illuminate synergies and trade-offs, which could
be critical for final judgement. Our example concerns the ecological
values of the Murray-Darling Basin; however, the framework is equally
applicable to social, economic, and cultural values and the toolkit
itself has a modular design to incorporate such response models. Our
framework drives a consistent approach to processing, scaling, analyses,
and visualising outcomes, with the flexibility built-in for most end
uses. Thus, HydroBOT provides a good avenue for informed decision-making
for water management in the Murray Darling Basin.

We expand upon the utility of an existing diver-indicator model by
linking its indicators \emph{EF: indicators same as values here I
suppose? I think cos this paper is quite dense and tackles a lot of
different types of variables and data visualisation techniques, it's
really important to use consistent terms.} to objectives. This increases
the transparency in the causal relationships that underpin the model and
builds understanding and trust in the outcomes.

concerning which adaptation options should be implemented under changing
climate by explicitly modelling effects on both ecological and other
values.

The development of HydroBOT is necessary to better understand the
impacts of climate change and the different adaptation options in
response to climate change, on water-dependent social, economic,
environmental and cultural values. It incorporates new and existing
information, knowledge and models to enable transparent, repeatable
assessments of impacts and adaptation to future climates. In summary,
the toolkit ingests scenarios (for example, climate or flow timeseries),
feeds them to a response model (such as the EWR tool), reports outcomes
and enables comparisons among scenarios. It uses the links in the
response models for visualisation of the complex inter-relationships
between water-dependent outcomes. This aids transparency and improves
communication of the outputs.

\emph{GEORGIA (priority ecological functions and environmental assets)}

\hypertarget{references}{%
\section{References}\label{references}}

\renewcommand{\bibsection}{}
\bibliography{references.bib}

\hypertarget{appendix-1}{%
\section{Appendix 1}\label{appendix-1}}

\hypertarget{sec-sec-glossary}{%
\subsection{Glossary}\label{sec-sec-glossary}}

\hypertarget{component-overview}{%
\subsubsection{Component overview}\label{component-overview}}

The HydroBOT architecture comprises three major components, the
Controller, Aggregator and Comparer, that receive data and information
from several input sources, including input scenarios, hydrological
modelling, spatial data, causal networks and response models
(Table~\ref{tbl-components}). Here we detail some specific details about
each of these components and other input sources.

\hypertarget{tbl-components}{}
\begin{longtable}[]{@{}
  >{\raggedright\arraybackslash}p{(\columnwidth - 6\tabcolsep) * \real{0.0260}}
  >{\raggedright\arraybackslash}p{(\columnwidth - 6\tabcolsep) * \real{0.1607}}
  >{\raggedright\arraybackslash}p{(\columnwidth - 6\tabcolsep) * \real{0.1492}}
  >{\raggedright\arraybackslash}p{(\columnwidth - 6\tabcolsep) * \real{0.6641}}@{}}
\caption{\label{tbl-components}Components of HydroBOT
architecture}\tabularnewline
\toprule\noalign{}
\begin{minipage}[b]{\linewidth}\raggedright
General Toolkit components
\end{minipage} & \begin{minipage}[b]{\linewidth}\raggedright
General component definitions
\end{minipage} & \begin{minipage}[b]{\linewidth}\raggedright
Specific components used in our example
\end{minipage} & \begin{minipage}[b]{\linewidth}\raggedright
Details of Specific components
\end{minipage} \\
\midrule\noalign{}
\endfirsthead
\toprule\noalign{}
\begin{minipage}[b]{\linewidth}\raggedright
General Toolkit components
\end{minipage} & \begin{minipage}[b]{\linewidth}\raggedright
General component definitions
\end{minipage} & \begin{minipage}[b]{\linewidth}\raggedright
Specific components used in our example
\end{minipage} & \begin{minipage}[b]{\linewidth}\raggedright
Details of Specific components
\end{minipage} \\
\midrule\noalign{}
\endhead
\bottomrule\noalign{}
\endlastfoot
Input data & Hydrologic data (timeseries). Typically representing
multiple scenarios, e.g.~climate and climate adaptations. May include
other inputs as needed by response models.~ & Modified historical
hydrographs to represent hypothetical climate change and adaptations &
Daily flow rates for NUMBER GAUGES for 15 scenarios \\
Controller & Interface between input data, response model, and other
toolkit components. Sets up run(s). & Sets up links to data and
parameters for EWR tool & NA \\
Response models & A model that contains are set of causal relationships
between a driver and responses, e.g.~social, cultural, environmental, or
economic values. & Specific response model = EWR tool & The EWR tool
holds databases of the EWRs required to meet the environmental
objectives of the basin, which protect or enhance environmental assets
that are valued based on ecological significance. \\
Aggregator & Aggregates response model results to scales across the
dimensions of time, space, and theme.~ & Response model sets the base
scale for aggregation. EWR tool assesses hydrologic indicators (theme)
at gauges (space) and internally aggregates over time. & The spatial
dimension consists of gauges nested within planning units within the
basin. \\
NA & NA & NA & The theme dimension consists of multiple EWRs (hydrologic
indicators) that apply to environmental objectives (many-to-one), which
apply to other levels in the causal network (e.g.~Objectives, Targets,
and Long-term targets). \\
Comparer & Compares scenarios (typically) or other groupings. Provides
standard outputs including comparison methods, plots, and tables. &
Comparison of environmental values at various theme scales for the
example climate and adaptation scenarios & - \\
Causal networks & Describe causal relationships between values. & Long
Term Water Plan (LTWP) & Plans required of Basin States by the
Murray-Darling Basin Plan. Long term water plans give effect to the
Basin-wide Environmental Watering Strategy relevant for each river
system and will guide the management of water over the longer term.
These plans will identify the environmental assets that are dependent on
water for their persistence, and match that need to the water available
to be managed for or delivered to them. The plan will set objectives,
targets and watering requirements for key plants, waterbirds, fish and
ecosystem functions. DPIE is responsible for the development of nine
plans for river catchments across NSW, with objectives for five, 10 and
20-year timeframes. ~ \\
\end{longtable}

\hypertarget{aggregation}{%
\subsubsection{Aggregation}\label{aggregation}}

Aggregation functions Aggregation functions determine outcomes and
reflect processes or values. Aggregating a set of outcomes yields a
different outcome depending on the function used. Thus, function choice
should be considered carefully and reflect the processes involved or
management goals. HydroBOT provides default functions for the mean,
compensating (e.g.~max), limiting (min), and spatially-weighted versions
of the same. It also provides the ability for the user to define any
desired function, allowing for more complex situations.

\hypertarget{spatial-aggregation}{%
\paragraph{Spatial aggregation}\label{spatial-aggregation}}

Maps allow large-scale visualization of ecological objectives under
various scenarios. These visualisations can be quite important for
communications and quickly grasping how outcomes aggregate spatially
(Figure~\ref{fig-spatial-scaling}) and spatial patterns in the data
(Figure~\ref{fig-gauge-to-sdl-map}). Management targets are often
defined spatially, and in the case of EWRs, they are defined in local
Planning Units (which are constituents of the SDL units defined above),
where outcomes may depend on hydrographs at one or more gauges.
Representing the outcomes as maps can provide intuitive assessment of
the condition of values across space and whether different spatial
scales or spatial locations are responding differently to scenarios.
Moreover, the ecology (in this example) or other processes might
themselves be large-scale, and so capturing the condition over a large
area is a better descriptor of the true outcome than assessing each
specific location separately. For example, objective EF3 is ``Provide
movement and dispersal opportunities for water dependent biota to
complete lifecycles and disperse into new habitats within catchments'',
and so necessarily incorporates a spatial dimension. Particularly in
these situations, the aggregation method should be considered carefully
-- for movement opportunities to succeed, perhaps the success of the SDL
unit should be determined by the lowest value at a gauge if it
represents a loss of connectivity. In contrast, for WB4: Increase
opportunities for colonial waterbird breeding, it might be sufficient if
a single site within the SDL unit provides those opportunities.

\begin{figure}

{\centering \includegraphics{demo_paper_files/figure-pdf/fig-spatial-scaling-1.pdf}

}

\caption{\label{fig-spatial-scaling}Environmental water requirements
(EWRs) are defined by hydrographs at gauges (a), and apply to Planning
Units (b). The outcomes at these scales of definition can then be
aggregated to larger spatial areas, such as SDL units (c), which we do
here with an area-weighted mean for the example EWR NF1 in the baseline
climate and adaptation scenario (E1, simple historical hydrograph with
no climate or adaptation changes).}

\end{figure}

\hypertarget{temporal-aggregation}{%
\paragraph{Temporal aggregation}\label{temporal-aggregation}}

\hypertarget{objective-aggregations}{%
\paragraph{Objective aggregations}\label{objective-aggregations}}

\hypertarget{sec-ewr-table}{%
\subsection{Response models}\label{sec-ewr-table}}

\hypertarget{ewrs}{%
\paragraph{EWRs}\label{ewrs}}

\emph{GEORGIA Need a table of their English definitions, the full EWR
table isn't helpful}

\begin{table}

\end{table}

\hypertarget{appendix-2}{%
\section{Appendix 2}\label{appendix-2}}

\hypertarget{sec-scenarios}{%
\subsection{Scenarios}\label{sec-scenarios}}

\begin{table}

\caption{\label{tbl-scenarios}Demonstration scenarios are a factorial
combination of `climate' (scaled flow) and `adaptation' (pulsed
additions). Climate scenarios included in this demonstration were
produced by applying a flow multiplier to historical flows. Adaptation
options were applied to each climate scenario with additional flows
added throughout the period of September to December. For ease of
display in some figures, we provide alpha codes for the `climate'
changes and numeric codes for the `adaptations'. Mostly frequently, we
present the `A' (0.5 x historical flow), `E' (1.0 x historical flow and
`I' (2.0 x historical flow) scenarios, with adaptions 1 (no flow
addition), 2 (addition of 250 ML/d) and 3 (addition of 6500
ML/d).}\begin{minipage}[t]{0.50\linewidth}
\subcaption{\label{tbl-scenarios-1}Climate}

{\centering 

\begin{tabular}[t]{lr}
\toprule
Climate code & Flow multiplier\\
\midrule
A & 0.50\\
B & 0.67\\
C & 0.80\\
D & 0.91\\
E & 1.00\\
F & 1.10\\
G & 1.20\\
H & 1.50\\
I & 2.00\\
\bottomrule
\end{tabular}

}

\end{minipage}%
%
\begin{minipage}[t]{0.50\linewidth}
\subcaption{\label{tbl-scenarios-2}Adaptation}

{\centering 

\begin{tabular}[t]{rr}
\toprule
Adaptation code & Flow addition (ML/d)\\
\midrule
1 & 0\\
2 & 250\\
3 & 6500\\
4 & 12000\\
\bottomrule
\end{tabular}

}

\end{minipage}%

\end{table}

\begin{verbatim}
Scale for y is already present.
Adding another scale for y, which will replace the existing scale.
\end{verbatim}

\begin{figure}

{\centering \includegraphics{demo_paper_files/figure-pdf/fig-hydrographs-1.pdf}

}

\caption{\label{fig-hydrographs}Hydrographs for two example gauges
(represented by colour) with `climate' scenarios on rows and
`adaptation' scenarios as columns. Scenario codes as in
Table~\ref{tbl-scenarios}.}

\end{figure}

\hypertarget{sec-baselining}{%
\subsection{Baselined hydrographs}\label{sec-baselining}}

\begin{figure}

{\centering \includegraphics{demo_paper_files/figure-pdf/fig-baseline-hydro-clim-1.pdf}

}

\caption{\label{fig-baseline-hydro-clim}Change in flow relative to the
baseline scenario. These are flat lines because the relativisation
occurs at each timepoint.The baseline scenario is represented by the
observed historical data with no climate change or adaptaion options
applied (scenario E1)}

\end{figure}

\begin{figure}

{\centering \includegraphics{demo_paper_files/figure-pdf/fig-baseline-hydro-adapt-1.pdf}

}

\caption{\label{fig-baseline-hydro-adapt}Change in flow volume compared
to the baseline scenario. This comparison is done using the difference,
and so represents the flow additions. The baseline scenario is
represented by the observed historical data with no climate change or
adaptaion options applied (scenario E1)}

\end{figure}

\hypertarget{sec-causalnetwork-versions}{%
\section{Appendix 3}\label{sec-causalnetwork-versions}}

\hypertarget{additional-causal-netwok-plots}{%
\subsection{Additional causal netwok
plots}\label{additional-causal-netwok-plots}}

\hypertarget{section}{%
\paragraph{}\label{section}}

\hypertarget{appendix-4}{%
\section{Appendix 4}\label{appendix-4}}

\hypertarget{additional-comparison-plots}{%
\subsection{Additional comparison
plots}\label{additional-comparison-plots}}

\hypertarget{sec-map-versions}{%
\subsubsection{Map aggregation}\label{sec-map-versions}}

The goal here is to show the values at the sdl scale, but also to show
how the gauges aggregate to that scale. In the text,
Figure~\ref{fig-gauge-to-sdl-map} has a few examples of SDL-scaled
outcomes, while Figure~\ref{fig-spatial-scaling} has the scaling for
one. Figure~\ref{fig-gauge-to-sdl-map-all} shows the SDL outcomes for
all ecological objectives contributing the priority ecological function,
along with the constituent gauges.

\begin{figure}

{\centering \includegraphics{demo_paper_files/figure-pdf/fig-gauge-to-sdl-map-all-1.pdf}

}

\caption{\label{fig-gauge-to-sdl-map-all}all panels version.}

\end{figure}

The full set of SDL units and ecological values

\begin{figure}

{\centering \includegraphics{demo_paper_files/figure-pdf/fig-smooth-all-1.pdf}

}

\caption{\label{fig-smooth-all}Smoothed fits to assess change in
performance across the `climate' scenarios. Points are individual
ecological objectives, fitted lines are loess smooths. Separate fits are
done for each adaptation option, and so differences between lines of
different colors represents the impact of those adaptations. Rows are
ecological values groupings, columns are SDL units. Note the very
different scale of the y-axis.}

\end{figure}

\hypertarget{heatmaps}{%
\subsubsection{Heatmaps}\label{heatmaps}}

\begin{figure}

{\centering \includegraphics{demo_paper_files/figure-pdf/fig-heatmap-1.pdf}

}

\caption{\label{fig-heatmap}Condition results visualised as a heatmap
with the two scenario definitions on the axes. This approach allows
visualising changes in outcome as the result of multiple axes on which
scenarios might differ. These axes might be different aspects of
scenario creation (as here), or they might be different axes describing
the outcome of scenarios (e.g.~two different hydrometrics such as mean
flow and flow variance)}

\end{figure}

\hypertarget{causal-networks}{%
\subsubsection{Causal networks}\label{causal-networks}}

The causal networks shown in the text have been subset for readability,
and baselined. Here, we have the ones they're based on. \emph{I've
turned these off for word because diagrammer hates word and I don't have
the energy to do a workaround}.

\begin{figure}

{\centering \includegraphics{demo_paper_files/figure-pdf/fig-full-networks-1.pdf}

}

\caption{\label{fig-full-networks}Subset of the causal network for gauge
419001 for all ecological objectives. The data is at the gauge scale for
the first and second columns (EWRs and ecological objectives) and at the
SDL unit scale for the third (ecological values). Thus, the final set of
nodes contain information from other gauges that are not shown here for
clarity. Colors represent the condition for the halving (a), baseline
(b), and doubling (c) `climate' scenario. Low condition is dark blue,
high is light yellow.}

\end{figure}

The same set of nodes as in the text, but the actual condition values,
rather than relativized.

\begin{figure}

{\centering \includegraphics{demo_paper_files/figure-pdf/fig-network-notbaseline-1.pdf}

}

\caption{\label{fig-network-notbaseline}Causal networks as in
\textbf{?@fig-network-subset}, but with the raw condition values for the
halved (a), baseline (b), and doubled (c) scenarios instead of
relativized to the baseline.}

\end{figure}

\begin{center}\rule{0.5\linewidth}{0.5pt}\end{center}

\hypertarget{journal-info}{%
\section{Journal info}\label{journal-info}}

\hypertarget{environmental-modelling-software}{%
\subsection{Environmental Modelling \&
Software}\label{environmental-modelling-software}}

• Author names and affiliations. Where the family name may be ambiguous
(e.g., a double name), please indicate this clearly. Present the
authors' affiliation addresses (where the actual work was done) below
the names. Indicate all affiliations with a lower-case superscript
letter immediately after the author's name and in front of the
appropriate address. Provide the full postal address of each
affiliation, including the country name and, if available, the e-mail
address of each author.

• Corresponding author. Clearly indicate who will handle correspondence
at all stages of refereeing and publication, also post-publication.
Ensure that telephone numbers (with country and area code) are provided
in addition to the e-mail address and the complete postal address.
Contact details must be kept up to date by the corresponding author.

• Present/permanent address. If an author has moved since the work
described in the article was done, or was visiting at the time, a
`Present address' (or `Permanent address') may be indicated as a
footnote to that author's name. The address at which the author actually
did the work must be retained as the main, affiliation address.
Superscript Arabic numerals are used for such footnotes.

\hypertarget{authorship}{%
\subsubsection{Authorship}\label{authorship}}

Authorship should be limited to those who have made a significant
contribution to the conception, design, execution, or interpretation of
the reported study. All those who have made significant contributions
should be listed as co-authors. Where there are others who have
participated in certain substantive aspects of the research project,
they should be acknowledged or listed as contributors. Acknowledgement
of the contributions of authors is encouraged (see Acknowledgements
section below). The corresponding author should ensure that all
appropriate co-authors and no inappropriate co-authors are included on
the paper, and that all co-authors have seen and approved the final
version of the paper and have agreed to its submission for publication.

Title, Authors, Affiliations and Contact details

\hypertarget{abbreviations}{%
\subsubsection{Abbreviations}\label{abbreviations}}

Define abbreviations that are not standard in this field in a footnote
to be placed on the first page of the article. Such abbreviations that
are unavoidable in the abstract must be defined at their first mention
there, as well as in the footnote. Ensure consistency of abbreviations
throughout the article.

\hypertarget{separate-files}{%
\subsection{Separate files}\label{separate-files}}

\hypertarget{highlights}{%
\subsubsection{Highlights}\label{highlights}}

Highlights are optional yet highly encouraged for this journal, as they
increase the discoverability of your article via search engines. They
consist of a short collection of bullet points that capture the novel
results of your research as well as new methods that were used during
the study (if any). Please have a look at the examples here: example
Highlights.

Highlights should be submitted in a separate editable file in the online
submission system. Please use `Highlights' in the file name and include
3 to 5 bullet points (maximum 85 characters, including spaces, per
bullet point).

Highlights are mandatory for this journal. They consist of a short
collection of bullet points that convey the core findings of the article
and should be submitted in a separate file in the online submission
system. Please use `Highlights' in the file name and include 3 to 5
bullet points (maximum 85 characters, including spaces, per bullet
point). See https://www.elsevier.com/highlights for examples.

\hypertarget{graphical-abstract-1}{%
\subsubsection{Graphical abstract}\label{graphical-abstract-1}}

A Graphical abstract is optional and should summarize the contents of
the article in a concise, pictorial form designed to capture the
attention of a wide readership online. Authors must provide images that
clearly represent the work described in the article. Graphical abstracts
should be submitted as a separate file in the online submission system.
Image size: Please provide an image with a minimum of 531 × 1328 pixels
(h × w) or proportionally more. The image should be readable at a size
of 5 × 13 cm using a regular screen resolution of 96 dpi. Preferred file
types: TIFF, EPS, PDF or MS Office files. See
https://www.elsevier.com/graphicalabstracts for examples.

\hypertarget{abstract-not-included-in-section-numbering}{%
\subsubsection{Abstract (not included in section
numbering)}\label{abstract-not-included-in-section-numbering}}

A concise and factual abstract is required, with a restriction of 150
words. The abstract should state briefly the purpose of the research,
the principal results and major conclusions. An abstract is often
presented separately from the article, so it must be able to stand
alone. For this reason, References should be avoided, but if essential,
then cite the author(s) and year(s). Also, non-standard or uncommon
abbreviations should be avoided, but if essential they must be defined
at their first mention in the abstract itself.

\hypertarget{keywords}{%
\subsubsection{Keywords}\label{keywords}}

Immediately after the abstract, provide a maximum of 6 keywords, using
American spelling and avoiding general and plural terms and multiple
concepts (avoid, for example, `and', `of'). Be sparing with
abbreviations: only abbreviations firmly established in the field may be
eligible. These keywords will be used for indexing purposes.

\hypertarget{software-andor-data-availability}{%
\subsubsection{Software and/or data
availability}\label{software-andor-data-availability}}

Most EMS papers should include a software/data availability section
containing as much of the following information as possible: name of
software or dataset, developer and contact information, year first
available, hardware required, software required, availability and cost.
Also for software: program language, program size; for data: form of
repository (database, files, spreadsheet), size of archive, access form.
Note that ``Contact the author'' is not acceptable for software or data
access. Please use online data and software storage and retrieval
systems such as GitHub, BitBucket, FigShare, HydroShare or others to
make your data and software readily available. Links to commercial
software and data access web sites are also acceptable.

When a software component is an essential part of the paper, authors
should be prepared to make it available to reviewers during the review
process. To preserve the anonymity of reviewers, the authors should make
the software available for a download, protecting it if needed by a
password that is communicated to the editors.




\end{document}
